\documentclass[12pt]{article}

\usepackage{tgschola}

\hyphenpenalty = 10000

\begin{document}

% \large

Good afternoon, ladies and gentlemen, my name is Roman Maksimovich and I will be representing Serbia in the section of mathematics with a research of convex spaces.\\

The concept of convexity is extremely popular throughout different fields of mathematics, such as geometry or functional analysis. However, quite little academic attention has been given to the \textit{abstraction} of this idea, which is what my work is about. Personally, I enjoy the beauty of the 'topology-style' proof methodology that is common for such theories and I am very excited to share with you my results.\\

We start with the definition of convex space, a set with a family of subsets satisfying the \textit{following} three axioms: it contains the space and the empty set, it is closed under intersections and closed under unions of nets. Now, the convex hull of a set is simply \textit{the smallest} convex set containing it.\\

An important object is the \textit{polytope}, which is the convex hull of a \textit{finite} set. In fact, a convexity is \textit{defined} by its polytopes, in the sence that a set is convex \textit{if and only if} it contains the polytope generated by every its finite subset. Polytopes possess some very useful properties. For example, you can \textit{clearly} identify the vertices of this polygon here. In general, those polytopes that have a \textit{unique} finite generating set, are called \textit{free}. The initial convex space is called \textit{free} if it contains only free polytopes. Then, we can define the \textit{dimension} of a polytope. For instance, this heptagon here is of dimension \textit{two}, since it fits within a triangle, but does not fit in a segment. The rule for \textit{higher} dimensions is the same. Furthermore, dimension can be generalized for an arbitrary set by comparing the local dimensions at each point.\\

Now we begin to see how \textit{geometrical} convexity is. A hyperplane is the union of a maximal net of codimensional polytopes. It possesses some technical qualities, such as the same dimension as that of the polytopes, or the fact that every hyperplane is generated by a unique net.\\

\newpage

The \textit{Polytope Union Lemma} states that if two polytopes both contain a polytope of the same dimension, then the dimension of their union is again the same. What is really \textit{fascinating} is that \textit{if and only if} this lemma holds, which is not always the case, then there is a natural equivalence between two ways of defining hyperplanes: through nets and by taking the union of all codimensional polytopes containing a given one. The Polytope Union Lemma plays a significant role in further research.\\



Now we examine the structures that can \textit{produce} convexity, starting from partial orders. We say that the convex hull of a set contains all the points bounded from both sides by its elements, as you can see on the picture. An interesting result is that every order convexity is free.\\

Next, we call a convex set \textit{linear} simply if its convexity is induced by a linear order. A convex space is called \textit{1-affine} if every its convex segment is linear. But what's important is that \textit{this idea} can be naturally generalized for higher dimensions, using only the terms of convexity. We call a space \(n\)-\textit{affine} if every \(n\)-dimensional hyperplane is divided in two special half-spaces by any its \((n-1)\)-dimensional subhyperplane. Affinity is a powerful geometrical property that will be useful.\\

Another inducing structure is the metric. With the notion of distance we can construct the segment between two points, and then define convexity traditionally as containing the segment connecting every pair of points. In general, convex spaces defined that way are called segmential.\\

\newpage

There are important concepts in the setting of segmential spaces. First is the join, where we connect a given point with every element of a given set and take the union of the segments. A convex space is called \textit{join-commutative} if taking the join of a point and a polytope is the \textit{same} as adding the point to the \textit{generating set} of the polytope. Second, consider this finite set of four points. It is clearly not convex, but how can we make it that way? Well, we just add the missing segments, and the result is the following skeleton. It is, however, \textit{still} not convex, so we \textit{repeat} the procedure. Now, if after a finite amount of steps we come to the actual convex hull of the original set, this is called finite-segmentiality. It follows that every 2-affine finite-segmential convex space, where the Polytope Union Lemma holds, \textit{must} be free.\\

Now we make a step aside to consider \textit{uniquely geodesic metric spaces}, in which every pair of points is connected by a unique distance-minimizing path, the length of which is equal to the \textit{distance} between the points. It follows that the image of such a path is precicely the metric segment connecting the points, and, what's more, every segment is a free polytope.\\


The Polytope Intersection Lemma basically states that the intersection of two polytopes is \textit{itself} a polytope, and that it possesses the same properties. It can be shown that in a free, finite-dimensional convexity, where both the Union and the Intersection lemmas hold, the interiors of polytopes form a base for a topology.\\

Now we generalize convexity to its local version, where the entire space may not be convex, but each point must have a convex neighborhood. This approach is supported by the intuition that sometimes, as with this shape on the slide, the small uniquely geodesic sets behave better than the overall space, which is not uniquely geodesic.\\

To see the applications of local convexity, consider the Eucledian plane and the standard open disk. You can see that on the plane any polytope fits within a triangle, whereas on the disk this is not always the case. Therefore, these convex spaces are not isomorphic, but locally they clearly represent the exact same convexity.\\

\newpage

Finally, we come to Riemannian geometry. Every Riemannian manifold is equipped with a convexity induced by the Riemannian metric. It is a known result that every such manifold is locally uniquely geodesic, which allows it to satisfy the convex properties discussed previously. Riemannian geometry is where the future of my work lies. Firstly, convexity is already a part of this already developed field, and secondly, we have seen that abstract convexity, with hyperplanes and affinity, is somewhat geometrical in nature. Intuitively, convexity is "in between" metric and topology, both of which are essential to Riemannian geometry. As a possibility, this may very well be the start of a new field, called \textit{convex geometry}.\\

Here is a summary of my results. The work is divided into \textit{three} branches, each giving a somewhat different approach to studying convexity. In internal theory we study \textit{the abstract} concepts, in inducing structures we consider \textit{particular} convex spaces and generalize their properties, in induced structures we see what convexity itself can produce. For this reason my work is more wide rather than deep, which makes it more applicable, as the results are more diverse. The last branch that deals with Riemannian manifolds is the least developed, but it has a big potential, as Riemannian geometry is a very popular field with many results to base a research upon.\\

Thank you all for your attention. I have spared much detail, so I'm looking forward to your questions.

\end{document}